\section{Dati}

I dati acquisiti dall'applicazione in tempo reale sono di tipo \textbf{time-series} o serie
temporali; si tratta di dati strettamente dipendenti da una coordinata temporale (ad esempio
il momento di acquisizione) e che presentano come conseguenza alcune peculiarità come il fatto
che spesso sono i dati recenti ad essere più "interessanti" e di conseguenza raramente si ha
la necessità di aggiornare dati passati.
\\
Ne sono alcuni esempio i dati acquisiti da sensori ad esempio metereologici, quelli relativi
all'utilizza delle risorse hardware di un personal computer o quotazioni di titoli finanziari.

\subsection{Dati finanziari}

I dati finanziari sono ricevuti dalla piattaforma di exchange di criptovalute Binance \cite{binance}
e comprendono i migliori prezzi di offerta e richiesta di un determinato simbolo e le rispettive
quantità; segue un esempio:

\begin{lstlisting}[language=json,firstnumber=1]
{
    "u": 5828881697,        // order book updateId
    "s": "BTCUSDT",         // symbol
    "b": "10262.83000000",  // best bid price
    "B": "1.88008400",      // best bid qty
    "a": "10262.94000000",  // best ask price
    "A": "6.48000500"       // best ask qty
}
\end{lstlisting}
%
I dati sono forniti in real-time senza nessun parametro temporale; di conseguenza verrà
aggiunto al momento della ricezione un campo "timestamp" con l'istante corrente come valore.
Durante l'analisi verrà principalmente utilizzato il valore "askprice".

\subsection{Tweets}

Anche i tweets possono essere considerati serie temporali data la notevole importanza
dell'istante di pubblicazione, soprattuto se analizzati per legati alla finanza.
\\
Sono ricevuti attraverso le API per sviluppatori fornite da Twitter \cite{twitter} e si
presentano come segue:

\begin{lstlisting}[language=json,firstnumber=1]
{
    "created_at": "Wed Sep 09 15:19:42 +0000 2020",
    "id": 1303714706687963136,
    "id_str": "1303714706687963136",
    "text": "... $50 ETH GIVEAWAY ...",
    "source": "...",
    "truncated": false,
    "in_reply_to_status_id": null,
    "in_reply_to_status_id_str": null,
    "in_reply_to_user_id": null,
    "in_reply_to_user_id_str": null,
    "in_reply_to_screen_name": null,
    "user": {
        ...
    }
    "geo": null,
    "coordinates": null,
    "place": null,
    "contributors": null,
    "retweeted_status": {
        ...
    }
    "is_quote_status": false,
    "quote_count": 0,
    "reply_count": 0,
    "retweet_count": 0,
    "favorite_count": 0,
    "entities": {
        ...
    }
    "favorited": false,
    "retweeted": false,
    "possibly_sensitive": false,
    "filter_level": "low",
    "lang": "en",
    "timestamp_ms": "1599664782579"
}
\end{lstlisting}
%
Si nota la presenza di ben 2 campi temporali ("created\_at" e "timestamp\_ms") che
rappresentano in realtà il medesimo istante temporale ma con precisione differente.
Tuttavia è possibile notare un certo ritardo nella ricezione dei tweets di circa 5-10
secondi, perciò è risultato necessario un campo aggiuntivo chiamato "received\_at" così
da tenere traccia del momento di ricezione.
L'analisi verrà successivamente effettuata principalmente sul campo "text" contente il
corpo del tweet.

Lo schema di entrambe le categorie di dati è costante per ogni messaggio rendedoli dati
strutturati o semi-strutturati per quanto riguarda i tweets, che presentano uno schema più complicato contente
elementi innestati (abbreviati nell'esempio per motivi di spazio) e del testo.
Nonostante ciò, anche per il fatto che dell'intero tweet i campi analizzati saranno il testo
e le coordinate temporali, è possibile utilizzare una rappresentazione relazionale.

% 2 parole sul fatto che sono strutturati

% sono timeseries